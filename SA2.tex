%!TEX TS-program =  pdflatex
\documentclass[a4paper,12pt,twoside]{scrreprt}
%%%%%%%%%%%%%%%%%%%%%%%%%%%%%%%%%%%%%%%%%%%%%%%%%%%%%%%%%%%%%%%%
% text encoding
%

\usepackage[latin1]{inputenc}

\usepackage[T1]{fontenc}
%\usepackage{ngerman,a4wide}
\usepackage[ngerman]{babel}
\usepackage{longtable}
\usepackage{color,listings,multicol}
\usepackage{float}
\usepackage[format=hang]{caption}
%usepackage[caption = false,format=hang]{subfig}
\captionsetup[subfloat]{justification=RaggedRight}
\usepackage[format=hang]{subfig}
%\captionsetup{format=hang}
\usepackage[numbers]{natbib}
\usepackage{mathtools}
\usepackage{fancyhdr}

\newcommand{\vect}[1]{\mathbf{#1}}
%\newcommand{\vect}[1]{\vec{#1}}


%f�r eps-Graphiken
%\DeclareGraphicsExtensions{.png,.pdf,.jpg,.mps,.eps}
%\DeclareGraphicsRule{.eps}{mps}{*}{}
\setlength{\headheight}{15pt}
%\bibliographystyle{natdin}  % put at beginning of document
%\usepackage{epspdfconversion}
%\usepackage{hyperref}


%%%%%%%%%%%%%%%%%%%%%%%%%%%%%%%%%%%%%%%%%%%%%%%%%%%%%%%%%%%%%%%%%%
\begin{document}

\setlength{\parindent}{0cm}


\author{Markus Sons}

\title{Schnittalgorithmus f�r XFEM mit der Level-Set-Methode}
\date{\today}

\maketitle

\tableofcontents


%%%%%%%%%%%%%%%%%%%%%%%%%%%%%%%%%%%%%%%%%%%%%%%%%%%%%%%%%%%%%%%%%%

%  Headings
\pagestyle{fancy}
\fancyhead{}  \fancyfoot{} % clear all fields
\renewcommand{\chaptermark}[1]{\markboth{#1}{}}
\renewcommand{\sectionmark}[1]{\markright{\thesection\space\space #1}}
\fancyhf{}
\fancyhead[LE,RO]{\thepage}
\fancyhead[LO]{\rm \rightmark}
\fancyhead[RE]{\rm \leftmark}
\fancyfoot[C]{\thepage}

\newpage
\addcontentsline{toc}{chapter}{Literaturverzeichnis}

%%%%%%%%%%%%%%%%%%%%%%%%%%%%%%%%%%%%%%%%%%%%%%%%%%%%%%%%%%%%%%%%%%

\chapter{Einleitung}
% FEM hat Vorteile gegen�ber FVM (Fehlerabsch�tzung)
% Verbrennung / Mehrphasenstr�mung spezielle Anforderung -> Sprung/knick in p und u
% Entweder sehr fein aufl�sen am Interface(fr�her), oder neue Ansatzfunktionen (XFEM)

%%%%%%%%%%%%%%%%%%%%%%%%%%%%%%%%%%%%%%%%%%%%%%%%%%%%%%%%%%%%%%%%%%
\chapter{Grundlagen}

\section{Navier-Stokes Gleichungen}

\subsection{Grundgleichung}
Die Navier-Stokes-Gleichung 

\subsection{Zweiphasenstr�mung und Oberfl�chenspannung}
% Besonderheiten der Zweiphasenstr�mung (-> warum XFEM?)

\section{Interface-Beschreibung}
Zur Beschreibung des Interfaces kann entweder eine Interface Tracking- oder eine Interface Capturing-Methode verwendet werden. Bei der Interface-Tracking-Methode wird das Interface explizit durch die Vernetzung beschrieben. Das Netz wird mit dem Interface weiterbewegt. Ein Problem dieser Methode ist, dass Topologie-�nderungen wie z.B. das Rekombinieren von zwei Blasen zu einer Gr��eren nicht dargestellt werden k�nnen.\\

Im vorliegenden Fall wird die Level-Set-Methode, eine Interface-Capturing-Methode genutzt.

\subsection{Level-Set-Methode}


\[\frac{\partial\phi}{\partial t} +  \vect u \cdot \nabla\phi = 0\]

\section{Diskretisierung}

%
%
\subsection{Standard-Galerkin FEM}

%
%
\subsection{eXtended Finite Element Method}
\subsection{Zeitintegration}

%%%%%%%%%%%%%%%%%%%%%%%%%%%%%%%%%%%%%%%%%%%%%%%%%%%%%%%%%%%%%%%%%%

\section{Schnittalgorithmus}

In BACI sind bereits zwei verschiedene Schnittalgorithmen implementiert. Der gew�nschte Algorithmus kann �ber den Parameter 

\section{Vorhandene Algorithmen}
\subsection{Tetgen}
\subsection{Hexahedra}

\section{Implementierter Algorithmus}
\subsection{Zerlegung in Tetraeder}
\subsection{Schnittf�lle}
\subsection{Verbesserungsm�glichkeiten}

%%%%%%%%%%%%%%%%%%%%%%%%%%%%%%%%%%%%%%%%%%%%%%%%%%%%%%%%%%%%%%%%%%

\chapter{Ergebnisse}
\section{Zalesaks-Disk}
\subsection{Massenverlust}
\subsection{Geometrieerhaltung}

\section{Collapsing Watercolumn}

%%%%%%%%%%%%%%%%%%%%%%%%%%%%%%%%%%%%%%%%%%%%%%%%%%%%%%%%%%%%%%%%%%

\chapter{Ausblick}

\appendix

\chapter{Code}

\end{document}

