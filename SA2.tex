%!TEX TS-program =  pdflatex
\documentclass[a4paper,12pt,twoside]{scrreprt}
%%%%%%%%%%%%%%%%%%%%%%%%%%%%%%%%%%%%%%%%%%%%%%%%%%%%%%%%%%%%%%%%
% text encoding
%
%For Macs use
%    \usepackage[applemac]{inputenc}
%For Unix Latin1 use
    \usepackage[latin1]{inputenc}
%For PC Codepage 850 use
%    \usepackage[cp850]{inputenc}
%For PC Codepage 437 use
%    \usepackage[cp437]{inputenc}
%For Windows ANSI use
%    \usepackage[ansinew]{inputenc}
%For Unicode use ???
%    \usepackage[utf8]{inputenc}
%\usepackage[applemac]{inputenc}

\usepackage[T1]{fontenc}
\usepackage{ngerman,a4wide}
\usepackage{longtable}
\usepackage{color,listings,multicol}
\usepackage{float}
\usepackage[format=hang]{caption}

%usepackage[caption = false,format=hang]{subfig}
\captionsetup[subfloat]{justification=RaggedRight}
\usepackage[format=hang]{subfig}
%\captionsetup{format=hang}
\usepackage[numbers]{natbib}

%f�r eps-Graphiken
%\DeclareGraphicsExtensions{.png,.pdf,.jpg,.mps,.eps}
%\DeclareGraphicsRule{.eps}{mps}{*}{}
\setlength{\headheight}{15pt}
\bibliographystyle{natdin}  % put at beginning of document
%\usepackage{epspdfconversion}
%\usepackage{hyperref}
\usepackage{fancyhdr}

%%%%%%%%%%%%%%%%%%%%%%%%%%%%%%%%%%%%%%%%%%%%%%%%%%%%%%%%%%%%%%%%%%
\begin{document}

\setlength{\parindent}{0cm}


\author{Markus Sons}

\title{Implementierung eines Schnitt-Algorithmus f�r die Level-Set-Methode und Untersuchung der kollabierenden Wassers�ule mit XFEM}
\date{\today}

\maketitle

\tableofcontents


%%%%%%%%%%%%%%%%%%%%%%%%%%%%%%%%%%%%%%%%%%%%%%%%%%%%%%%%%%%%%%%%%%

%  Headings
\pagestyle{fancy}
\fancyhead{}  \fancyfoot{} % clear all fields
\renewcommand{\chaptermark}[1]{\markboth{#1}{}}
\renewcommand{\sectionmark}[1]{\markright{\thesection\space\space #1}}
\fancyhf{}
\fancyhead[LE,RO]{\thepage}
\fancyhead[LO]{\rm \rightmark}
\fancyhead[RE]{\rm \leftmark}
\fancyfoot[C]{\thepage}

\newpage
\addcontentsline{toc}{chapter}{Literaturverzeichnis}

%%%%%%%%%%%%%%%%%%%%%%%%%%%%%%%%%%%%%%%%%%%%%%%%%%%%%%%%%%%%%%%%%%

\chapter{Einleitung}

\chapter{Grundlagen}

Das ist nur ein kleiner Test damit ich das Programm verstehe.

Schifffahrt

\section{Fluid}

Kleiner Test

\section{Finite Elemente Methode}

\bibliography{bibData}

\end{document}

