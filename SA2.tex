%!TEX TS-program =  pdflatex
\documentclass[a4paper,12pt,twoside]{scrreprt}
%%%%%%%%%%%%%%%%%%%%%%%%%%%%%%%%%%%%%%%%%%%%%%%%%%%%%%%%%%%%%%%%
% text encoding
%
%For Macs use
%    \usepackage[applemac]{inputenc}
%For Unix Latin1 use
    \usepackage[latin1]{inputenc}
%For PC Codepage 850 use
%    \usepackage[cp850]{inputenc}
%For PC Codepage 437 use
%    \usepackage[cp437]{inputenc}
%For Windows ANSI use
%    \usepackage[ansinew]{inputenc}
%For Unicode use ???
%    \usepackage[utf8]{inputenc}
%\usepackage[applemac]{inputenc}

\usepackage[T1]{fontenc}
\usepackage{ngerman,a4wide}
\usepackage{longtable}
\usepackage{color,listings,multicol}
\usepackage{float}
\usepackage[format=hang]{caption}

\newcommand{\vect}[1]{\boldsymbol{#1}}

%usepackage[caption = false,format=hang]{subfig}
\captionsetup[subfloat]{justification=RaggedRight}
\usepackage[format=hang]{subfig}
%\captionsetup{format=hang}
\usepackage[numbers]{natbib}

%f�r eps-Graphiken
%\DeclareGraphicsExtensions{.png,.pdf,.jpg,.mps,.eps}
%\DeclareGraphicsRule{.eps}{mps}{*}{}
\setlength{\headheight}{15pt}
%\bibliographystyle{natdin}  % put at beginning of document
%\usepackage{epspdfconversion}
%\usepackage{hyperref}
\usepackage{fancyhdr}

%%%%%%%%%%%%%%%%%%%%%%%%%%%%%%%%%%%%%%%%%%%%%%%%%%%%%%%%%%%%%%%%%%
\begin{document}

\setlength{\parindent}{0cm}


\author{Markus Sons}

\title{Implementierung eines Schnitt-Algorithmus f�r die Level-Set-Methode und Untersuchung der kollabierenden Wassers�ule mit XFEM}
\date{\today}

\maketitle

\tableofcontents


%%%%%%%%%%%%%%%%%%%%%%%%%%%%%%%%%%%%%%%%%%%%%%%%%%%%%%%%%%%%%%%%%%

%  Headings
\pagestyle{fancy}
\fancyhead{}  \fancyfoot{} % clear all fields
\renewcommand{\chaptermark}[1]{\markboth{#1}{}}
\renewcommand{\sectionmark}[1]{\markright{\thesection\space\space #1}}
\fancyhf{}
\fancyhead[LE,RO]{\thepage}
\fancyhead[LO]{\rm \rightmark}
\fancyhead[RE]{\rm \leftmark}
\fancyfoot[C]{\thepage}

\newpage
\addcontentsline{toc}{chapter}{Literaturverzeichnis}

%%%%%%%%%%%%%%%%%%%%%%%%%%%%%%%%%%%%%%%%%%%%%%%%%%%%%%%%%%%%%%%%%%

\chapter{Einleitung}

\chapter{Grundlagen}

\section{Navier-Stokes Gleichung}



\subsection{Grundgleichung}
% Navier-Stokes-Gleichung (Herleitung?)

\subsection{Zweiphasenstr�mung und Oberfl�chenspannung}
% Besonderheiten der Zweiphasenstr�mung (-> warum XFEM?)


\section{Level-Set-Methode}
Zur Beschreibung des Interfaces kann entweder die "'Interface Tracking'"- oder die "'Interface Capturing'"-Methode verwendet werden. Bei der Interface-Tracking-Methode wird das Interface explizit durch die Vernetzung beschrieben. Das Netz wird mit dem Interface weiterbewegt. Ein Problem dieser Methode ist, dass Topologie-�nderungen wie z.B. das Rekombinieren von zwei Blasen zu einer Gr��eren nicht dargestellt werden k�nnen.\\

Beim "Interface Capturing" hingegen wird das Interface implizit beschrieben. Im vorliegenden Fall wird die Level-Set-Methode verfolgt, bei der das Interface durch die Null-Iso-Linie der Level-Set-Funktion beschrieben wird. Zus�tzlich erf�llt die Level-Set-Funktion zus�tzlich die "signed distance property", der Betrag der Funktion entspricht also an jedem Punkt den Abstand zum Interface, w�hrend das Vorzeichen die Seite darstellt.

\[ \frac{\partial\phi}{\partial t} +  \vect u \cdot \nabla\phi = 0\]

\section{Diskretisierung}

%
%
\subsection{Standard-Galerkin FEM}

%
%
\subsection{eXtended Finite Element Method}
\subsection{Zeitintegration}

\chapter{Schnittalgorithmus}

In BACI sind bereits zwei verschiedene Schnittalgorithmen implementiert. Der gew�nschte Algorithmus kann �ber den Parameter 

\section{Vorhandene Algorithmen}
\subsection{Tetgen}
\subsection{Hexahedra}

\section{Implementierter Algorithmus}
\subsection{Zerlegung in Tetraeder}
\subsection{Schnittf�lle}
\subsection{Verbesserungsm�glichkeiten}

\chapter{Ergebnisse}
\section{Zalesaks-Disk}
\subsection{Massenverlust}
\subsection{Geometrieerhaltung}

\section{Collapsing Watercolumn}

\chapter{Ausblick}

\appendix

\chapter{Code}

\end{document}

